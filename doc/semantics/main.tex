\documentclass[12pt]{article}
\usepackage{a4wide}
\usepackage{amsmath,amssymb, latexsym, stmaryrd}
\usepackage{tensor}
\usepackage{braket}


\def\loc{\alpha}
\def\sig{s}
\def\unit{()}
\def\susp{\downarrow}
\def\present{{\bf{1}}}
\def\absent{{\bf{0}}}
\newcommand{\parl}[2]{\tensor[_{#1}]{\parallel}{_{#2}}}
\newcommand{\vect}[1]{\overline{#1}}
\newcommand\envs[1]{\langle #1 \rangle}
\def\env{\rho}
\def\sigenv{\theta}
\def\heap{\sigma}
\def\continue{{\mathcal{C}}}
\def\ret{{\mathcal{R}}}
\def\suspend{{\mathcal{S}}}
\def\variable{\tt{var}}
\def\dom{{\mathit{dom}}}
\def\lhs{{\mathit{lhs}}}
\def\allocate{{\tt{ref}}}
\def\case{{\tt{case}}}
\newcommand{\ite}[3]{{\tt{if}}~#1~#2~#3}
\newcommand{\while}[2]{{\tt{while}}~#1~#2}
\def\emit{{\tt{emit}}}
\newcommand{\when}[2] {{\tt{when}}~#1~#2}
\newcommand{\watching}[2]{{\tt{watching}}~#1~#2}
\newcommand{\eval}[3]{{\mathit{eval}}(#1,#2,#3)}
\newcommand{\evalLeft}[3]{{\mathit{eval_L}}(#1,#2,#3)}
\newcommand{\evalRight}[3]{{\mathit{eval_R}}(#1,#2,#3)}
\newcommand{\field}[2]{#1\!:\!#2}
\def\sail{{\sc{sail}}}
\def\integer{{\tt{int}}}
 \def\bool{{\tt{bool}}}
 \def\float{{\tt{float}}}
 \def\char{{\tt{char}}}
 \def\str{{\tt{string}}}
 \def\reftype{\mathit{ref}}
 \def\boxtype{\mathit{box}}
 \def\arraytype{\mathit{array}}
 \def\struct{{\tt{struct}}}
\def\enum{{\tt{enum}}}
\def\method{{\mathit{method}}}
\def\activate{{\mathit{enter}}}
\def\deactivate{{\mathit{exit}}}
\def\emptyoffset{\epsilon}
\def\skipp{{\tt{skip}}}
\def\return{{\tt{return}}}
\newcommand{\block}[2]{\{#1\}_{#2}}
\def\updateValue{{\mathit{update_V}}}
\def\value{v}
\def\values{{\mathit{Val}}}
\def\address{a}
\def\addresses{\mathcal{A}}
\def\location{\ell}
\def\locations{{\mathit{Loc}}}
\def\signal{s}
\def\signals{{\mathit{Sig}}}
\def\variables{{\mathit{Var}}}
\newcommand{\valarray}[1]{A(#1)}
\newcommand{\valstruct}[1]{\{#1\}}
\newcommand{\valenum}[2]{S_{#1}(#2)}
\def\paths{{\mathit{Path}}}
\def\expressions{{\mathit{Expr}}}
\def\stacks{{\mathcal{E}}}
\def\heaps{{\mathcal{H}}}
\def\undef{{\mathit{undefined}}}
\def\offset{{\mathit{o}}}
\def\offsets{{\mathit{Off}}}
\def\stopp{{\tt{stop}}}
\def\signal{{\tt{signal}}}
\def\dsignal{{\tt{signal}}}
\def\kill{{\mathit{kill}}}
\def\collect{{\mathit{collect}}}
\newcommand{\addressOf}[1]{#1\lightning}

\def\suspended{{\mathit{suspended}}}
\def\unlock{{\mathit{unlock}}}


\title{Core \sail}
\begin{document}
\maketitle
\section{Overview}
The \sail{} language belongs to the family of Synchronous Reactive Programming
(or SRP for short) languages. It is a domain specific language aiming at increasing the
reliability of reactive programming, especially in the field of IOT.

In \sail{}, reactivity is expressed through parallel composition and signal broadcast. Parallel composition of commands
$c_1$ and $c_2$ is noted $c_1 \parallel c_2$. It relies on a cooperative scheduling.
The execution of a program consists in successive runs named {\emph{instants}}.
A signal named $s$ is declared through the command $\signal~s$. Its lifetime is the innermost surrounding block of code.
It is {\emph{emitted}} through the command $\emit~s$. It can also be produced by the environment
between instants. Once emitted, a signal is {\emph{present}} for the duration of the current instant.
The execution of a command may be subject to the presence of a signal.
This is expressed by the statement $\when{s}{c}$.
For example, the following command prints a message if $s$ is present. Otherwise it is suspended
until the signal is emitted or a preemption occurs (see the ${\tt{watching}}$ construct below)
$$
  \begin{array}{l}
    {\tt{pause}} = \when{s}{{\mathit{print\_string}}("{\text{s was here}}")}
  \end{array}
$$
The end of an instant occurs when  no further progress can be made. Either because the execution
is terminated or because all components are waiting for an absent signal.
Every component of a program has the opportunity to react to a present signal before the instant
terminates, components have a consistent view of their environment.
Finally, a command may be preempted at the end of the current instant if it is blocked and if a given signal is present.
This is expressed by the statement $\watching{s}{c}$ which behaves as $c$ but preempts its residual at the end
of the instant if $s$ is present. If $c$ terminates during the instant then $\watching{s}{c}$ also terminates.
As an example, consider the following statement which we will note ${\tt{pause}}$ in the rest of the document.
$$
  \begin{array}{c}
    \signal~s; \watching{s}{\{\emit~s;\signal~s';\when{s'}{}\}}
  \end{array}
$$
The ${\tt{pause}}$ statement suspends its execution until the end of the instant and resumes at the next instant.
Ths signal $s'$ is not emitted and thus the program is suspended because of the ${\tt{when}}$ statement. As the signal $s'$ is
local, it can't be emitted whatever is the context of the program. At the end of
the instant the signal $s$ is present and thus the ${\tt{watch}}$ statement is preempted.

The following example shows a possible use of this construct.
$$
  \begin{array}{l}
    \signal~s;                                      \\
    \signal~s';                                     \\
    \{                                              \\
    \qquad {\mathit{print\_string}}("A");           \\
    \qquad \emit~s;                                 \\
    \qquad {\tt{pause}};                            \\
    \qquad \emit~s'                                 \\
    \parallel                                       \\
    \qquad \when{s'}{{\mathit{print\_string}}("B")} \\
    \parallel                                       \\
    \qquad \watching{s}{                            \\\qquad\qquad\when{s'}{{\mathit{print\_string}}("C")}}\\
    \}
  \end{array}
$$
The program prints "A" at the first instant and "B" at the second instant. The message "C" is never printed
because the {\tt{watching}} block is preempted.

\section{Syntax}

The \sail{} language supports usual ground data types (boolean, integers, floating points numbers,
characters and strings) as well as compound data types (arrays, structures and enumerations) and generic data types.
Types in \sail{} are defined by :
$$
  \tau ::= \bool \mid \integer \mid  {\tt{float}} \mid \char \mid \str \\
  \mid \arraytype \langle\tau\rangle \mid \reftype\langle\tau\rangle \mid \boxtype \langle \tau \rangle \mid id\langle \tau,\ldots,\tau\rangle \mid A
$$
where $\arraytype\langle\tau\rangle$ denotes an array of values of type $\tau$, $\boxtype\langle\tau\rangle$
denotes a pointer to a heap allocated value of type $\tau$, $\reftype\langle\tau\rangle$ denotes a shared reference and
$id\langle \tau,\ldots,\tau \rangle$ denotes a user type which is either a structure or an enumeration.
A type variable $A$ denotes a generic type. Closed types are types in which no type variable occurs.
Structure and enumeration are respectively defined by :
$$
  \begin{array}{lcl}
    \struct~id\lceil\langle A,\ldots,A \rangle\rceil~\{f:\tau,\ldots,f:\tau\}                                                      \\
    \enum~id\lceil\langle A, \ldots A\rceil\rangle~\{f\lceil :(\tau,\ldots,\tau) \rceil,\ldots,f\lceil :(\tau,\ldots,\tau)\rceil\} \\
  \end{array}
$$
where $\lceil . \rceil$ denotes an optional element.
In both cases, the type variables may occur in the types used in the definition.
As an example, the type of generic list may be defined in \sail{} by
$$
  \begin{array}{l}
    \enum~{\mathit{option}}<A>\{                                                                                                      \\
    \qquad None,                                                                                                                      \\
    \qquad Some (A)                                                                                                                   \\
    \}                                                                                                                                \\
    \struct~{\mathit{list}}<A>\{                                                                                                      \\
    \qquad {\mathit{head}} : option \langle \reftype\langle {\mathit{node}\langle A\rangle}\rangle \rangle \\\} \\
    \struct~{\mathit{node}}\langle A\rangle \{                                                                                        \\
    \qquad elem : A,                                                                                                                  \\
    \qquad next : option \langle \reftype\langle {\mathit{node}\langle A\rangle}\rangle \rangle                                       \\
    \}
  \end{array}
$$
Expressions of the {\sail} language are defined by the following grammar :
$$
  \begin{array}{lcl}
    p&::=& x \mid p.f \mid *p\\
    e & ::= & \mid p \mid c \mid \varominus e \mid e \varoplus e \mid \{\field{f}{e},\ldots,\field{f}{e}\}
    \mid C(e,\ldots,e) \mid \&e \mid {\mathit{box}}(e)
  \end{array}
$$
Variables are names ranging over a finite set and are noted $x$, $y$, $z$ and so on.
A constant $c$ is a literal denoting a boolean, an integer, a floating-point value, a character or a string.
Given a literal $c$, we note $\widehat{c}$ the corresponding value.
Unary and binary  operators (respectively $\varominus$ and $\varoplus$) are usual operators over integer and
boolean value and are assumed to come with semantics functions (noted $\widehat{\varominus}$
and $\widehat{\varoplus}$).
An expression $[e_0,\ldots,e_{n-1}]$ denotes an array value filled
with the values denoted by $e_0,\ldots,e_{n-1}$ and
and $e_1[e_2]$ denotes the value at the position denoted by $e_2$ of such a value
denoted by $e_1$.
An expression $\{\field{f_0}{e_0},\ldots, \field{f_{n-1}}{e_{n-1}}\}$ denotes a
structured value filled with values $e_0,\ldots,e_{n-1}$ and $e.f$ denotes the
value at position $f$ of such a value denoted by $e$.
An expression $\&e$ denotes the memory location at which the value denoted by $e$ is stored.
An expression $*e$ denotes the value stored at the memory location denoted by $e$.

Commands of Core-\sail{} contain usual commands such as variable declaration, sequential composition,
conditional, loops and method calls. In Core sail, method calls return no values.
Core-\sail{} also contains reactive constructs for parallel composition and signal handling.
The grammar of commands is given below, followed by an intuitive description for the more significative commands.
In next section, we will see that the execution stack will reflect the parallel nature of terms. We choose
to distribute the stack over commands (symbols $\omega$). This will be explained latter.
For now the symbols $\{c\}_\omega$ can be read as the command $c$.
$$
  \begin{array}{lcl}
    c & ::= & \mid \variable~x:\tau \mid \signal~\sig \mid \skipp \mid e_1 = e_2    \\
      &     & \mid \{c\}_\omega \mid c;c \mid \ite{e}{c}{c} \mid \while{e}{c} \mid \case~e~\{p:c,\ldots,p:c\}
    \mid m(e,\ldots,e) \mid \return
    \\
      &     & \mid \emit~s \mid \when{s}{\{c\}_\omega} \mid \watching{s}{\{c\}_\omega}
    \mid \block{c}{\omega} \parallel \block{c}{\omega}
  \end{array}
$$
\begin{itemize}
  \item $\variable~x:\tau$ and $\signal~s$ declare respectively a variable $x$
        of type $\tau$ and a signal $s$
  \item $e_1 = e_2$ stores the value of $e_1$ at the memory location denoted by $e_1$
  \item the sequence, conditional, loop commands and method calls behave as usual.
  \item $\case~e~\{p:c,\ldots,p:c\}$ performs pattern matching on the value denoted by
        $e$. It takes the first pattern $p$ of the list that match the value and behaves as the
        corresponding command $c$ in an environment augmented by a mapping of the variable of $p$
        to the appropriate values.
  \item $\when{e}{\block{c}{\omega}}$ behaves as $c$ when the signal $s$ is present. When $s$ is absent the
        command is suspended.
  \item $\watching{s}{\block{c}{\omega}}$ behaves as $c$ but, if $s$ is present at the end of the instant the
        whole block terminates
  \item $\block{c_1}{\omega_1} \parallel \block{c_2}{\omega_2}$ runs $c_1$ and $c_2$ in parallel.
        Parallel composition terminates when both branches terminate.
\end{itemize}



\begin{paragraph}{Methods and programs}
  The syntax of method declaration is
  $$
    {\tt{method~id\langle A,\ldots,A \rangle(x:\tau,\ldots,x:\tau)}}~c
  $$
  A program in Core-\sail{} is given by a sequence of declarations of user-defined types and methods. It also provide an entry
  point which is a single command (which corresponds to the Main process in \sail{}). Consider the following example which computes
  the factorial of $5$.
  $$
    \begin{array}{l}
      {\tt{method}}~{\mathit{factorial}}(x : int, y:\&int) \{           \\
      \qquad {\tt{if}} (x==0~{\tt{or}}~x==1 ) \{*y = 1; {\tt{return}}\} \\
      \qquad{\tt{else}} \{                                              \\
      \qquad\qquad  {\tt{var}}~z : {\tt{int}};                          \\
      \qquad\qquad  {\mathit{factorial}}(x - 1, \&z);                   \\
      \qquad\qquad  *y = x * z;                                         \\
      \qquad\qquad  {\tt{return}}                                       \\
      \qquad\}                                                          \\
      \}
      \\\\

      {\tt{var}}~x : int;                                               \\
      {\mathit{factorial}} (5, \&x);                                    \\
    \end{array}
  $$
\end{paragraph}

\begin{paragraph}{Memory locations}
  We assume a built-in method
  $$box\langle A\rangle (x:A, y:\&box\langle A\rangle)$$
  which allocates a new memory location, of type $box\langle A\rangle$
  which receives the value of $x$. The new memory location is written in $y$. Core-\sail{} distinguish two kind of
  memory locations. Those of type $box\langle\tau\rangle$, for some type $\tau$, are memory locations explicitly allocated
  by the user as in $x = box(1,\&y)$. Memory locations of type $ref\langle\tau\rangle$ are shared references obtained by
  the $\&$ operator.
\end{paragraph}

\section{Dynamic semantics}
Core \sail{} commands execute in the context of an environment and a heap. The environment maps program
variables and signal names to memory addresses (noted $\alpha$ possibly with subscripts).
The heap maps memory addresses to values and signal states.

\subsection{Semantics domain}
Let $\addresses$ be a countable set of adresses, noted $\address$ and let $\offsets$ be the set of offsets defined 
by $\offset ::= \emptyoffset \mid f;\offset$.
The set $\values{}$ of \sail{} values, noted $\value{}$, is defined by the grammar below:
$$
  \begin{array}{lcl} v & ::= & c \mid \{\field{f}{v}, \ldots, \field{f}{v}\} \mid C(v,\ldots,v)
            \mid \address \mid \address_{\offset,m} \mid \bot
  \end{array}
$$
where $\{\field{f_i}{v_i}\}_{i=1}^n$ is a structured value and $C(v_0,\ldots,v_{n-1})$ is an enum value.
Values of the form $\address$ and $\address_\offset$ are named memory location.
An offset $\offset$ is a sequence of field names. A mutability status is an element of $\{r,w\}$.
A memory location $\address_{\offset,m}$ denotes a borrowing of the memory location $\address$ at offset $\offset$
which is either mutable ($m=w$) or immutable ($m=r$).
The value $\bot$ denotes a moved value.
We note $\value_\offset$ for the value of $\value$ at offset $\offset$ and 
$\value[\value']_\offset$ the update of $\value$ with $\value'$ at $\offset$, which is recursively defined on $\offset$ by
$$
  \begin{array}{lcllcl}
    v_\emptyoffset &=& \value & \qquad
    \{\dotso,\field{f_{i}}{v},\dotso\}_{f_i;o} &=&\value_\offset\\
    v[v']_{\emptyoffset} &=& v' & \qquad
    \{\dotso,\field{f_{i}}{v},\dotso\}[v']_{f_i;o} &=& 
    \{\dotso,\field{f_{i}}{v}[v']_o,\dotso\}
  \end{array}
$$
A scope $\omega$ is a partial function from variables to addresses and 
frame $\env$ is a non empty list of scopes. The state of signals is represented by two possible values $\absent$ and $\present$.
A Heap, noted $\heap$, is a partial mapping from addresses to values and signal states.
$$
\begin{array}{c}
\omega \in \variables \rightharpoonup \addresses  \qquad \env \in (\variables \rightharpoonup \addresses)^+\qquad
\heap \in \locations \rightharpoonup \values^\bot \cup\{\absent, \present\}
\end{array}
$$
We note $l \cdot t$ the list $l$ augmented with the element $e$. 
We note $\dom(f)$ the domain of a function $f$. Given two function $f$ and $g$, we note $f \uplus g$ 
the disjoint union of $f$ and $g$. We note $f[d \mapsto c]$ the function that maps $d$ to $c$ and $d'$ to $f(d)$ for all $d'\not =  d$.
Function application is extended to, possibly empty, list of functions as follows : 
$$
\epsilon(x) {\text{ is undefined}} \qquad 
(\overline{f}\cdot g)(x) = g(x) {\text{ if $x \in \dom(g)$ and }} \overline{f}(x) {\text{ otherwise}}
$$ 
where $\epsilon$ denotes the empty list.


\subsection{Semantics of expressions}
We note $\offsets$ the set of offsets, noted $\offset$, defined by
$$
  o ::= \epsilon \mid f;o
$$
We note $c$ the value corresponding to the litteral $c$ and $\ominus, \oplus$ an arbitrary operator.
Each operator is assumed to have a semantics, using the same notation.
We define the set $\locations$ of memory locations, noted $\location$ as $\locations = \addresses \times \offsets$.
The semantics of expressions is devised in two functions
$$
  \begin{array}{lcl}
    {\mathit{eval_L}} & : & \paths{} \times \stacks{} \times \heaps{} \rightharpoonup \locations{}\\
    {\mathit{eval}}   & : & \expressions{} \times \stacks{} \times \heaps{}\rightharpoonup \values{} \times \heaps
  \end{array}
$$

$$
  \begin{array}{c}
    \cfrac{\env(x) = \address}
    {\evalLeft{x}{\env}{\heap} = (\address, \epsilon)}
    \qquad
    \cfrac{\evalLeft{p}{\env}{\heap} = (\address,o)}
    {\evalLeft{p.f}{\env}{\heap} = (\address, o.f)}
    \\\\
    \cfrac{\evalLeft{p}{\env}{\heap} = (\address_\star,o) \qquad \sigma(\address)[o] = (\address_\star',o')}
    {\evalLeft{*p}{\env}{\heap} = (\address_\star',o')}
    \\\\
    \cfrac{\evalLeft{p}{\env}{\heap} = (\address_\star,o) \qquad \sigma(\address)[o] = v \qquad Move(v)}
    {\eval{p}{\env}{\heap} = v, \heap[\address.o \mapsto \bot]}
    \\\\
    \cfrac{\evalLeft{p}{\env}{\heap} = (\address_\star,o) \qquad \sigma(\address)[o] = v \qquad Copy(v)}
    {\eval{p}{\env}{\heap} = v, \heap}
    \\\\
    \cfrac{}{\eval{c}{\env}{\heap} = c,\heap}
    \qquad
    \cfrac{\eval{e}{\env}{\heap} = v, \heap'}
    {\eval{\varominus e}{\env}{\heap} = \varominus v, \heap'}
    \\\\
    \cfrac{
        \eval{e_1}{\env}{\heap} = v_1,\heap'' \qquad 
        \eval{e_2}{\env}{\heap''} = v_2, \heap'}
    {\eval{e_1 \varoplus e_2}{\env}{\heap} = v_1 \varoplus v_2, \heap'}
    \\\\
    \cfrac{

      \eval{e_k}{\env}{\heap_k} = v_k,\heap_{k+1},~k\in \set{1,\ldots,n} \qquad 
      \sigma =\sigma_1 \qquad \sigma'=\sigma_{n+1}
    }
    {
      \eval{\valstruct{\field{f_i}{e_i}}_{i=1}^n}{\env}{\heap} = \valstruct{\field{f_i}{v_i}}_{i=1}^n, \heap'
    }
    \\\\
    \cfrac{
    \eval{e_k}{\env}{\heap_k} = v_k,\heap_{k+1}~k\in \set{1,\ldots,n} \qquad 
    \sigma =\sigma_1 \qquad \sigma'=\sigma_{n+1}
    }
    {\eval{C(e_1,\ldots,e_n)}{\env}{\heap} = C(v_1,\ldots,v_n) }
    \\\\
    \cfrac{\evalLeft{p}{\env}{\heap} = (\address, \offset)}
    {\eval{\&p}{\env}{\heap} = \address_o,\heap}
    \qquad{}
    \cfrac{
        \eval{e}{\env}{\heap'} = (v,\heap'') \qquad{
        (a,\heap') = {\mathit{fresh}}(\heap'') 
        }
    }
    {\eval{{\mathit{box}(e)}}{\env}{\heap} =  a,\heap'[a.\epsilon \mapsto v]}
  \end{array}
$$


\subsection{Semantics of commands}
The semantics of commands is divided in several levels. First we have microsteps which are given
by rules of the form $c,\rho,\sigma \rightarrow K, \omega,\heap'$. In a microstep, each component
is executed one time until it is terminated or suspended.
The status $K \in\{\continue, \suspend~c, \ret\}$ tells us if the command is terminated ($\continue$), suspended
with the continuation $c$ to resume ($\suspend~c$) or returning control to the caller ($\ret$).
The frame $\omega$ denotes the part of the stack allocated during the micro-steps. Commands containing blocks
of the form $\{.\}_\omega$ is a suspened command in which $\omega$ was recorded when the command suspends.
As mentioned earlier, we use these annotations to distribute the stack over commands.
\begin{figure}
  $$
    \begin{array}{c}
      \cfrac{x \not \in \dom (\rho) \qquad (\address,\heap') = {\mathit{fresh}}(\heap)}{

        \variable~[mut]~x:\tau, \env,\heap \rightarrow
        \continue, [x\mapsto \alpha],\heap'_{\address \leadsto {\mathit{undef}}}
      }
      \qquad
      \cfrac{s \not \in \dom(\rho) \qquad (\address,\heap') = {\mathit{fresh}}(\heap)}{
        \signal~\sig, \env, \heap\rightarrow
        \continue, [\sig \mapsto \address], \heap'_{\address \mapsto \absent}
      }
      \\
      \cfrac{}{\skipp,\env ,\sigma \rightarrow \continue, \varnothing,\sigma}
      \qquad{}
      \cfrac{}{\stopp,\env ,\sigma \rightarrow \continue, \varnothing,\sigma}
      \\\\
      \cfrac{
        \begin{array}{c}
          \evalLeft{p}{\env}{\heap} = (\address,o) \qquad
          \eval{e}{\env}{\heap} = (v, \heap')\\
          \heap(a.o) = v' \qquad \heap'' = {\mathit{drop}}(\heap',v')
        \end{array}
      }{
        p := e, \env,\heap \rightarrow
        \continue, \varnothing,\heap'[\address.o \mapsto v]
      }
    %   \\\\
    %   \cfrac{
    %     \begin{array}{c}
    %       \evalLeft{e_1}{\env}{\heap} = (\alpha,o) \qquad
    %       \evalRight{e_2}{\env}{\heap} = v \\
    %       \heap(\alpha) = v_\alpha \qquad \alpha' \not \in \dom(\heap)
    %       \qquad v' = \updateValue(v_\alpha,p,\alpha')
    %     \end{array}
    %   }{
    %     e_1 := {\tt{box}}(e_2), \env,\heap \rightarrow
    %     \continue, \varnothing,\heap_{\alpha' \leadsto v, \alpha \mapsto v'}
    %   }
      \\\\
      \cfrac{
        c_1, \env \cdot \omega,\heap \rightarrow \continue, \omega_1,\heap''\qquad
        c_2, \env \cdot \omega\omega_1,\heap'' \rightarrow \mathcal{K},\omega_2,\heap'
      }{
        c_1;c_2,\env \cdot \omega ,\heap \rightarrow \mathcal{K},\omega_1\omega_2,\heap'
      }
      \\\\
      \cfrac{
        c_1, \env,\heap\rightarrow \suspend~c_1', \omega,\heap'
      }{
        c_1;c_2,\env,\heap \rightarrow \suspend~c_1';c_2,\omega,\heap'
      }
      \qquad
      \cfrac{
        c_1, \env,\heap \rightarrow \ret, \omega,\heap'
      }{
        c_1;c_2,\env,\heap \rightarrow \ret,\omega,\heap'
      }
      \\\\
      \cfrac{
      c, \env \cdot \omega, \heap \rightarrow \suspend~c', \omega', \heap'
      }{
      \{c\}_{\omega},\env, \heap \rightarrow \suspend~\{c'\}_{\omega\omega'}, [~],\heap'
      }
      \\\\
      \cfrac{
      c, \env \cdot \omega, \heap \rightarrow
      \mathcal{K}, \omega', \heap' \qquad {\mathit{values}}(\omega\omega') = \overline{v}
      \qquad \mathcal{K} \in \continue,\ret \qquad
      \heap'' = drop(\heap', \overline{v})
      }{
      \{c\}_{\omega},\env, \heap \rightarrow \mathcal{K}, \varnothing,\heap''
      }
      \\\\
      \cfrac{
        \eval{e}{\env}{\heap} = ({\tt{true}}, \heap')
        \qquad \{c_1\}, \env,\heap' \rightarrow \mathcal{K},\omega,\heap''
      }{
        \ite{e}{\{c_1\}}{\{c_2\}},\env,\heap \rightarrow \mathcal{K},\omega,\heap''
      }
      \\\\
      \cfrac{
        \evalRight{e}{\env}{\heap} = ({\tt{false}}, \sigma') \qquad \{c_2\}, \rho,\heap'
        \rightarrow \mathcal{K},\omega,\heap'
      }{
        \ite{e}{\{c_1\}}{\{c_2\}},\env,\sigma \rightarrow \mathcal{K},\omega,\sigma'
      }
      \\\\
      \cfrac{
        \evalRight{e}{\env}{\heap} = {\tt{true}}\qquad \{c\}_\varnothing; \while{e}{c},\env,\heap \rightarrow \mathcal{K}, \omega,\heap'
      }{
        \while{e}{c}, \env, \heap \rightarrow \mathcal{K},\omega,\heap'
      }
      \\\\
      \cfrac{
        \evalRight{e}{\env}{\heap} ={\tt{false}}
      }{
        \while{e}{c}, \env, \heap \rightarrow \continue,\varnothing,\heap
      }
    \end{array}
  $$
  \caption{Microsteps}
  \label{fig:sem1}
\end{figure}
\begin{figure}
  $$
    \begin{array}{c}
      \cfrac{
      \begin{array}{c}
        \evalRight{e_i}{\env}{\heap} \Downarrow v_i \qquad 0\leq i \leq n-1                                                                                 \\
        {\tt{method}}~m\langle A_0,\ldots,A_{m-1}\rangle(\field{x_0}{\tau_0},\ldots,\field{x_{n-1}}{\tau_{n-1}})\lceil :\tau\rceil~c \in {\mathit{Methods}} \\
        \{c\}_{[x_i \mapsto v_i]_{i=0}^{n-1}},\epsilon, \heap \rightarrow \ret(v), \omega, \heap'
      \end{array}
      }{
      m(e_0,\ldots,e_{n-1}), \env, \heap \rightarrow \continue, \omega, \heap'
      }
      \\\\
      \cfrac{
        \begin{array}{c}
          \evalRight{e}{\env}{\heap} = v \qquad
          filter(p,v) = {\mathit{undef}} \\
          {\tt{case}}~e~\{\overline{p:c}\}, \env ,\heap \rightarrow \mathcal{K},\omega, \heap'
        \end{array}
      }{
        {\tt{case}}~e~\{p:c,\overline{p:c}\}, \env ,\heap \rightarrow \mathcal{K},\omega, \heap'
      }
      \\\\
      \cfrac{
        \begin{array}{c}
          \evalRight{e}{\env}{\heap} = v \qquad
          filter(p,v) = [(x_i,v_i)]_{i=0}^{n-1}              \\
          \{\ell_i\}_{i=0}^{n-1} \cap \dom(\heap)= \emptyset \\
          \{c\}_{[x_i \mapsto \ell_i]_{i=0}^{n-1}}, \env, \heap_{\{\ell_i, \mapsto v_i\}_{i=0}^{n-1}}
          \rightarrow \mathcal{K},\omega,\heap'
        \end{array}
      }{
        {\tt{case}}~e~\{p:c,\overline{p:c}\}, \env ,\heap \rightarrow \mathcal{K}, \omega , \heap'
      }
      % A PRIORI OMEGA = VIDE ICI
      % CE QUI EST LE CAS A CHAQUE FOIS QUE L'ON INSERE UN BLOC, FAIRE DEUX TYPES DE REGLES BLOCK OU NON BLOCK
      % CONTEXTE OOU NON CONTEXTE
      % NOTE {C} for {C}_[]
      \\\\
    \end{array}
  $$
  \caption{Microsteps}
  \label{fig:sem2}
\end{figure}
\begin{figure}
  $$
    \begin{array}{c}
      \cfrac{
        \begin{array}{c}
          \env(s) = \ell \qquad \sigma(\ell) = \present
          \qquad c,\env\cdot \varnothing,\heap,\rightarrow \mathcal{K},\omega',\heap'
          \qquad \mathcal{K} \in \continue,\ret \\
          \omega\omega' \lightning \overline{\alpha} \qquad
          \sigma'' = {\mathit{drop}}(\omega,\overline{\alpha})
        \end{array}
      }{
        \when{s}{\{c\}_\omega},\env,\heap\rightarrow
        \mathcal{K},\varnothing,\heap''
      }
      \\\\
      \cfrac{
      \env(s) = \ell \qquad \sigma(\ell) = \present \qquad
      c,\env \cdot \omega,\heap \rightarrow \suspend~c',\omega',\heap'
      }
      {
      \when{s}{\{c\}_\omega},\env,\heap\rightarrow
      \suspend~\when{s}{\{c'\}_{\omega\omega'}},\varnothing,\heap'
      }
      \\\\
      \cfrac{
      \env(s) = \ell \qquad \sigma(\ell)= {\bf{0}}
      }{
      \when{s}{\{c\}_\omega},\env,\heap \rightarrow
      \suspend~\when{s}{\{c\}_\omega},\varnothing,\heap
      }
      \\\\
      \cfrac{
        \begin{array}{c}
          c,\env \cdot \omega,\heap \rightarrow \mathcal{K},\omega',\heap'
          \qquad \mathcal{K} \in \continue,\ret \\
          \omega\omega' \lightning \overline{\alpha} \qquad
          \sigma'' = {\mathit{drop}}(\omega,\overline{\alpha})
        \end{array}
      }{
        \watching{s}{\{c\}_\omega},\env,\heap\rightarrow \mathcal{K}, \varnothing,\heap''
      }
      \\\\
      \cfrac{
      c,\env \cdot \omega,\heap\rightarrow \suspend~c',\omega',\heap'
      }{
      \watching{s}{\{c\}_\omega},\env,\heap \rightarrow \suspend~\watching{s}{\{c'\}_{\omega\omega'}},\varnothing,\heap'
      }
      \\\\
      \cfrac{
      \begin{array}{c}
        c_1, \env \cdot \omega_1, \heap \rightarrow
        \continue ,\omega_1' , \heap''
        \qquad
        c_2, \env \cdot \omega_2 , \heap'' \rightarrow
        \continue ,\env \cdot \omega_2' ,\heap'''
        \\
        \omega_1\omega_2\omega_1'\omega_2' \lightning \overline{\alpha} \qquad \heap''={\mathit{drop}}(\heap, \overline{\alpha})
      \end{array}
      }{
      \{c_1\}_{\omega_1} \parallel \{c_2\}_{\omega_2}, \env ,\heap \rightarrow
      \continue, \varnothing,\heap''
      }
      \\\\
      \cfrac{
      \begin{array}{c}
        c_1, \env \cdot \omega_1, \heap \rightarrow
        \suspend(c_1') ,\omega_1' , \heap''
        \qquad
        c_2, \env \cdot \omega_2 , \heap'' \rightarrow
        \suspend(c_2') ,\omega_2' ,\heap'
      \end{array}
      }{
      \{c_1\}_{\omega_1} \parallel \{c_2\}_{\omega_2}, \env ,\heap \rightarrow
      \suspend~\{c_1\}_{\omega_1\omega_1'} \parallel \{c_2\}_{\omega_2\omega_2'}, \varnothing, \heap'
      }
      \\\\
      \cfrac{
      \begin{array}{c}
        \{c_1\}, \env \cdot \omega_1, \heap \rightarrow
        \suspend(c_1') ,\omega_1' , \heap''
        \qquad
        \{c_2\}, \env \cdot \omega_2 , \heap'' \rightarrow
        \continue ,\omega_2' ,\heap'''
      \end{array}
      }{
      \{c_1\}_{\omega_1} \parallel \{c_2\}_{\omega_2}, \env ,\heap \rightarrow
      \suspend~\{c_1'\}_{\omega_1\omega_1'} \parallel \{\skipp\}_{\omega_2\omega_2'}, \varnothing,\heap'
      }
      \\\\
      \cfrac{
        \begin{array}{c}
          c \equiv c' \qquad c',\env,\heap \rightarrow K',\omega,\heap' \qquad K' \equiv K
        \end{array}
      }{
        c, \env ,\heap \rightarrow
        K,\omega,\heap'
      }
    \end{array}
  $$
  \caption{Microsteps}
  \label{fig:sem3}
\end{figure}
$$
  \begin{array}{c}
    \cfrac{}{c \equiv c} \qquad
    \cfrac{}
    {
    \{c_1\}_{\omega_1} \parallel \{c_2\}_{\omega_2} \equiv
    \{c_2\}_{\omega_2} \parallel \{c_1\}_{\omega_1}
    }
    \\\\
    \cfrac{}{K \equiv K} \qquad
    \cfrac{c \equiv c'}{\suspend~c \equiv \suspend~c'}
  \end{array}
$$

\def\kill{{\mathit{kill}}}
\def\collect{{\mathit{collect}}}
\newcommand{\addressOf}[1]{#1\lightning}

\def\suspended{{\mathit{suspended}}}
\def\unlock{{\mathit{unlock}}}

\subsection{steps and instants}

The execution of an instant is a succession of microsteps. We perform microsteps until
the command is either terminated or suspended (predicate $\suspended$).If the
command is suspended, we perform preemption (function $\unlock$) before to run a new instant.
The computation of instants is given in figure \ref{fig:instants}.
$$
  \begin{array}{c}
    \cfrac{
      \suspended(c,\env \cdot \omega,\heap)
    }{
      \suspended(\{c\}_\omega,\env,\heap)
    }
    \qquad
    \cfrac{
      \suspended(c_1,\env,\heap)
    }{
      \suspended(c_1;c_2,\env,\heap)
    }
    \\\\
    \cfrac{
      \env(s)=\alpha \qquad
      \heap(\alpha)=\absent \vee (\heap(\alpha)=\present \wedge
        {\mathit{suspended}}(c,\env \cdot \omega,\heap))
    }
    {{\mathit{suspended}}(\when{s}{\{c\}_\omega},\env,\heap)}
    \\\\
    \cfrac{
      \suspended(c,\env\cdot \omega,\heap)
    }{
      \suspended(\watching{s}{\{c\}_\omega},\env,\heap)
    }
    \\\\
    \cfrac{
      \suspended(c_1,\env \cdot \omega_1,\heap) \vee
      \suspended(c_2,\env \cdot \omega_2,\heap)
    }{
      \suspended({c_1} \parl{\omega_1}{\omega_2} {c_2}, \env, \heap)
    }
  \end{array}
$$
The function $\unlock$ terminates preemption blocks if the watched signal is present.
$$
  \begin{array}{lcll}
    \unlock(\{c\}_\omega, \env,\sigma)                      & = & \{c'\}_\omega                      & {\text{where }} c'=\unlock(c,\env\cdot\omega,\heap)                                              \\
    \unlock(c_1;c_2, \env,\sigma)                           & = & c_1';c_2                           & {\text{where }} c_1'=\unlock(c_1,\env,\heap)                                                     \\
    \unlock(\when{s}{\{c\}_\omega})                         & = & \when{s}{\{c'\}_\omega}            & {\text{where }} c'=\unlock(c,\env\cdot\omega,\heap)                                              \\
    \unlock(\watching{s}{\{c\}_\omega})                     & = & \skipp                             & {\text{if $\rho(s)=\alpha$ and $\sigma(\alpha)=\present$}}                                       \\
    \unlock(\watching{s}{\{c\}_\omega})                     & = & \watching{s}{\{c'\}_\omega}        & {\text{if $\rho(s)=\alpha$, $\sigma(\alpha)=\absent$ and $c'=\unlock(c,\env\cdot\omega,\heap)$}} \\
    \unlock(c_1 \parl{\omega_1}{\omega_2} c_2, \env, \heap) & = & c_1'\parl{\omega_1}{\omega_2} c_2' & {\text{where }} c_i' = \unlock(c_i,\env\cdot\omega_i,\heap),~i=1,2
  \end{array}
$$

We note $(c,\heap) \rightarrow (c', \heap)$ for $(c,\epsilon,\heap) \rightarrow (c',\varnothing,\heap')$,
$\suspended(c,\heap)$ for $\suspended(c,\varnothing,\heap)$ and
$\unlock(c,\heap)$ for $\unlock(c,\varnothing,\heap)$.
$$
  \begin{array}{c}
    \cfrac{
      (c,\heap) \rightarrow (\continue,\heap')
    }{
      (c,\heap) \Rightarrow \heap'
    }
    \qquad
    \cfrac{
      (c,\heap) \rightarrow (\suspend~c',\heap') \qquad
      \suspended (c',\heap')
    }{
      (c,\heap) \Rightarrow (c', \heap')
    }
    \\\\
    \cfrac{
      (c,\heap) \rightarrow (\suspend~c',\heap') \qquad
      \neg \suspended (c',\heap') \qquad (c',\heap') \Rightarrow (c'',\heap'')
    }{
      (c,\heap) \Rightarrow (c'', \heap'')
    }
  \end{array}
$$
\begin{figure}
  $$
    \begin{array}{c}
      \cfrac{
        (c,\heap) \Rightarrow \heap'
      }{
        (c,\heap) \Longmapsto \heap'
      }
      \qquad
      \cfrac{
        (c,\heap) \Rightarrow (c',\heap')
      }{
        (c,\heap) \Longmapsto (c', \heap')
      }
      \\\\\
      \cfrac{
        (c,\heap) \Longmapsto (c', \heap') \qquad \unlock(c',\heap) = c'' \qquad (c'', \heap') \Longmapsto (c''',\heap')
      }{
        (c,\heap) \Longmapsto (c''', \heap')
      }
    \end{array}
  $$
  \label{fig:instants}
  \caption{Instants}
\end{figure}
\begin{paragraph}{Execution of a program}
  Given a command $c$, the initial state is $\{c\}_\varnothing,\epsilon,h_\emptyset$
\end{paragraph}

%Note that $\{\}$ are not parentheses which are useless in an abtract grammar.


\end{document}
